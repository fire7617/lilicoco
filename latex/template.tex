%!TEX TS-program = xelatex
%!TEX encoding = UTF-8 Unicode

\documentclass[12pt, fleqn]{article}
\usepackage[a4paper, left=1.5cm, right=1.5cm, top=1.5cm, bottom=2.3cm]{geometry}

\usepackage{fontspec}
\usepackage{listings}

%% 圖片引用相關設定
\usepackage{graphicx}
\graphicspath{{./images/}}

%% 使用xCJK設定中文環境
\usepackage{xCJK}
\usepackage{CJKspace}
\setCJKmainfont[ItalicFont={DFHeiStd-W7}, BoldFont={DFHeiStd-W7}]{DFMingStd-W5}
%\setCJKfamilyfont{Japanese}{ヒラギノ角ゴ Pro W3}

%% 以下調整字距
\renewcommand\CJKglue{\hskip -0.3pt plus 0.08\baselineskip}
%% 以下調整行距,兩者 latex2html 皆不支援。
\linespread{1.2}
%% 以下表示各段落不縮排
\parindent=0pt
%% 縮小 quote 為正常行距,並使用較小字體
\renewenvironment{quote}
  {\list{}{\topsep 0ex\parsep 0ex\setlength\leftmargin{1.5em}%
           \rightmargin\leftmargin}\item\relax\linespread{1.0}\small}%
  {\endlist}

%% 由於把行距加大,要把腳注的行距縮成預設的,這個設定採自吳聰敏教授的
%% 《cwTeX 排版系統》修訂二版一書,頁 143。
\let\oldfootnote\footnote
\renewcommand\footnote[1]{\oldfootnote{\renewcommand\baselinestretch{1.0}%
\large\footnotesize\ignorespaces#1}}
\addtolength{\footnotesep}{3pt}

\XeTeXlinebreaklocale "zh"
\XeTeXlinebreakskip = 0pt plus 1pt 


\begin{document}
\begin{CJK*}{UTF8}{}
    \title{\textbf{軟體的特性為何}}
    \author{efang}
    \date{\today}
    \maketitle


\begin{lstlisting}[title={hello.cpp}, frameround=tfff, columns=fullflexible, numberstyle=\tiny,stepnumber=5, frame=single, basicstyle={\ttfamily \small}, numbers=left, commentstyle={\sffamily}, language=c++]
#include <iostream>
int main(int argc, char** argv){
    std::cout << "Hello World" << std::endl; // print ``Hello World''
}
\end{lstlisting}

\begin{lstlisting}[title={hello.cpp},
                   frameround=tfff,
                   columns=fullflexible,
                   numberstyle=\tiny,
                   stepnumber=5,
                   frame=single,
                   basicstyle={\ttfamily \small},
                   numbers=left,
                   commentstyle={\sffamily},
                   language=c++]
#include <iostream>
int main(int argc, char** argv){
    std::cout << "Hello World" << std::endl; // print ``Hello World''
}
\end{lstlisting}
この~FAQ~リストは、よくある質問とその答を集め、役に立つようにしたもので


\end{CJK*}
\end{document}
