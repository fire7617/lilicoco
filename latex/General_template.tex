%!TEX TS-program = xelatex
%!TEX encoding = UTF-8 Unicode
\documentclass[t,compress, hyperref={colorlinks,linkcolor=white,urlcolor=DarkBlue}]{beamer}
\usepackage{beamerthemesplit}
\usepackage{fontspec}

% 圖片引用相關設定
\usepackage{graphicx}
\graphicspath{{./images/}}

%佈景主設定
\usecolortheme{whale}
\usecolortheme{orchid}
%\usecolortheme{seagull}
\useoutertheme[footline=authortitle]{miniframes}
%\useoutertheme{infolines}


\definecolor{DarkBlue}{RGB}{0, 0, 139}
\definecolor{MediumBlue}{RGB}{0, 0, 205}
\definecolor{LightBlue}{RGB}{173, 216, 230}
\definecolor{PigmentBlue}{RGB}{51, 51, 153}

\setbeamercolor{palette primary}{bg=PigmentBlue!90}
\setbeamercolor{palette secondary}{bg=DarkBlue!120}
\setbeamercolor{frametitle}{fg=structure.fg!110, bg=white}
\setbeamercolor{section in head/foot}{parent=palette secondary}
\setbeamercolor{subsection in head/foot}{parent=palette primary}
\setbeamercolor{pagenumber in head/foot}{bg=PigmentBlue!30}

%\setbeamercolor{block title}{bg=OxfordRed!90}
%\setbeamercolor{block body}{bg=OxfordPastelRed!20}

\newcommand{\jj}{\vskip2ex}
\newcommand{\kk}{\vskip-2ex}
\newcommand{\from}{\leftarrow}
\newcommand{\us}{\underline{~}}
\newcommand{\nil}{[\kern1pt]}
\newcommand{\listr}[1]{[\,#1\,]}
\newcommand{\reason}[1]{\quad\{\,\text{#1}\,\}}
\newcommand{\conv}{\text{\u{}}}


\mode<presentation>{%
\newfontinstance\chfont{DFHeiStd-W5}
\newcommand{\chf}[1]{{\chfont #1}}          % 使用\chf{} 即可使用中文
% 整體文件設定
\linespread{1.10}    % 行距為原來的1.10倍
}

\mode<article>{%
\newfontinstance\chfont{DFMingStd-W5}
\newcommand{\chf}[1]{{\chfont #1}}          % 使用\chf{} 即可使用中文
}

% 大標題(副標題)指令
\newcommand{\bigtitles}[2]{\begin{frame}\begin{columns}\begin{column}{\textwidth}\begin{center}{\Huge #1}\\#2\end{center}\end{column}\end{columns}\end{frame}}


\title{Java Technicalities}
\subtitle{From a C/C\raisebox{0.7pt}{++} Programmer's View}
\author{yen3}
\institute{CGUCSIE}
\date{April 8, 2009}

\begin{document}

\setbeamertemplate{frametitle}
{\vskip.15cm
\hbox{\begin{beamercolorbox}[wd=\paperwidth,sep=.5cm]{frametitle}\usebeamerfont{frametitle}\insertframetitle\end{beamercolorbox}}\vskip-.45cm
}

\setbeamertemplate{footline}
{%
\leavevmode%
\hbox{\begin{beamercolorbox}[wd=.6\paperwidth,ht=2.5ex,dp=1.125ex,leftskip=.3cm,rightskip=.3cm plus1fill]{title in head/foot}%
\usebeamerfont{title in head/foot}\insertshorttitle
\end{beamercolorbox}%
\begin{beamercolorbox}[wd=.4\paperwidth,ht=2.5ex,dp=1.125ex,leftskip=.3cm,rightskip=.3cm]{pagenumber in head/foot}%
\usebeamerfont{pagenumber in head/foot}\hfill\insertframenumber/\inserttotalframenumber
\end{beamercolorbox}}%
\vskip0pt%
}



\frame{\titlepage}

\section{About}
\begin{frame}
    \frametitle{About Author}
    \begin{itemize}
        \item Computer Science Student
        \item Blog: \href{http://yen3rc.blogspot.com}{\underline{No title, no thinking, no meaning}}
        \item E-mail: yen3rc \chf{在} gmail \chf{答康}
        \item \chf{隨手書寫生活}
        \item C, C\raisebox{0.7pt}{++}, Java, Haskell, \LaTeX
    \end{itemize}
\end{frame}

\section[Outline]{}
\frame{\tableofcontents}

\begin{frame}
    \frametitle{About Slide}
    \begin{itemize}
        \item \chf{專題程式開發介紹相關簡報第二彈,陸續有其他介紹}
        \item \chf{感謝\href{http://www.csie.cgu.edu.tw/~cllee/}{\underline{李春良老師}}專題指導,使得這一系列簡報得以誕生}
        \item \chf{感謝} Josh Ko \href{http://joshkos.blogspot.com}{\underline{(Joshsoft)}}  \chf{在主題與技術上的指導與協助(還有英文)} XD
    \end{itemize}
\end{frame}

\subsection{Any Problem?}
\begin{frame}
    \begin{columns}[t]
        \begin{column}{\textwidth}
            \begin{center}
                {\Huge Do you have any problem ?}\\
                I am glad you have listened the slides from start to now. XD
            \end{center}
        \end{column}
    \end{columns}
\end{frame}

\end{document}
